\documentstyle[rhultechreport]{article}
\title{{\tt gt} quick guide}
\author{Adrian Johnstone}
\begin{document}
\makecstitle
\section{Introduction}
{\tt gt} is a standalone tool for transforming grammars in EBNF
into BNF. There can be no canonical translation from EBNF to BNF since
different parser generators demand grammars with different
chacteristics. The main goal of {\tt gt} is to allow generally
available published grammars (especially grammars from language
standards documents) to be transformed into grammars suitable for use
with standard tools in a way that is semi-automatic {\em and}
documented. 

\subsection{The need for flexible translation}
In general, EBNF constructs such as Kleene closure (the
'star' operator in regular expressions, or curly-braces in Wirth-style
grammars) can be translated into BNF in several ways, and 
it turns out that the best translation to use certainly
depends on the expected use of the grammar and may depend on
particular local features of the grammar. For instance, recursive descent
parser generators cannot usually accept left recursive definitions,
since left recursion leads to non-termination, yet bottom up parser
generators show performance improvements if left recursion is used in
preference to right recursion.

\section{Overall flow}



\appendix
\section{An {\tt rdp} grammar for {\tt gt}}
{\footnotesize
\begin{verbatim}
\end{verbatim}
}
\end{document}
