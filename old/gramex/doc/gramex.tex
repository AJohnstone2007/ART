\documentclass[11pt]{article}
\newcommand{\tech}[1]{{\em #1}\marginpar{\em #1}}
\newcommand{\lp}{\tt\em}
\newcommand{\lt}{\tt}
\newcommand{\us}{\_\,}
\newcommand{\bnfexp}{$\mbox{::= }$}
\newcommand{\bnfor}{$|$\ }

\usepackage{rhultechreport,epsfig,moreverb}
\title{{\lt gramex} - a tool for extracting grammar rules from typeset documents}
\author{Adrian Johnstone}
\begin{document}
\makecstitle
\thispagestyle{empty}
\vspace*{1cm}
\begin{center}\large\bfseries\sf Abstract\end{center}
In the C language community there is a tradition of using a minimalist
grammar specification format that is easy for humans to read, but not
directly suitable for input to common parser generators. {\lt gramex}
is a tool that reads plain-ASCII versions of these grammar
specifications and outputs a more familiar Bison-like version.  {\tt
gramex} has been used to extract grammars from the Kernighan and
Ritchie ANSI-C book, the ANSI C++ standard and the Java Language
Specification; and a special {\em Pascal} mode can be used to extract
rules from the Pascal standards documents. An accompanying tool {\lt
gramconv} can be used to convert {\lt gramex} generated files into
other formats, and to perform certain grammar translations such as
converting extended BNF into plain BNF.

\vspace*{3ex}

\noindent {\lt gramex} been built, run and tested using the Free
Software Foundation's GNU {\lt g++} compiler version 3.3.3 running
under Cygwin on Windows-XP.

\vspace*{\fill}
\begin{center}\large\bfseries\sf Typographical conventions\end{center}
We use {\lt this font} to represent literal input fragments and {\lp
this font} to represent input fragments which should be substituted
with an application specific literal text.

\vspace*{\fill}
\begin{center}
\fbox{\parbox{\textwidth}{This document is \copyright\,Adrian Johnstone
2006, 2007, 2011.\\[1ex]
Please send comments and error reports
to the address on the title page or electronically
to {\tt A.Johnstone@rhul.ac.uk}.
}}
\end{center}
\clearpage
\setcounter{page}{0}
\pagenumbering{arabic}
\section{Introduction}

In the C language community there is a tradition of using a minimalist
grammar specification format that is easy for humans to read, but not
directly suitable for input to common parser generators.  In standards
documents and some text books, different fonts are used to distinguish
terminals and nonterminals, but this formatting information is lost in
plain-ASCII versions of the documents. In addition, plain-English
comments are sometimes used to specify hard-to-capture constraints on
valid character and terminal sequences.

{\lt gramex} is a tool that reads plain-ASCII versions of these
grammar specifications and outputs a Bison-like~\cite{Bison}
version. {\tt gramex} has been used to extract grammars from the
Kernighan and Ritchie ANSI-C book, the ANSI C++ standard and the Java
Language Specification.  An accompanying tool {\lt
gramconv}~\cite{gramconv} can be used to convert {\lt gramex}
generated files into other formats, and to perform certain grammar
translations such converting extended BNF into plain BNF.

\section{Downloading and using {\tt gramex}}
The {\lt gramex} source code is a single file {\tt gramex.c} which may
be downloaded from the RHUL Compiler Group's website under
\verb+http://www.cs.rhul.ac.uk+.

{\lt gramex} is written in ANSI-C and compiles under the Free Software
Foundation's GNU {\lt g++} compiler using 
\begin{quote}
\verb|g++ -ansi -pedantic gramex.c|
\end{quote}

Input to {\tt gramex} should be an ASCII file containing the plain
text version of the rules. Run the tool with a command like:

\begin{quote}
{\lt gramex} {\lp options} {\lp sourcefile}
\end{quote}

where {\lp sourcefile} is the name of the plain ASCII input, and
{\lp options} may be zero or more of:
\begin{quote}
\begin{description}
\item[{\lt -c}]  suppress comment (non-rule) lines
\item[{\lt -e}]  treat [ ] and $\{ \}$ as EBNF meta symbols
\item[{\lt -i}]  treat productions indented by more than four spaces
\item[{\lt -p}]  process Pascal standard file (implies -e)
as continuations
\end{description}
\end{quote}

The translated output is sent to the console: use the output
redirection operator (\verb+>+) to capture it to a file:
\begin{quote}
\verb+gramex -c myfile.raw > myfile.gex+
\end{quote}

\section{Rule extraction}
Here is a fragment from the grammar included in the ANSI C++ standard.
\begin{quote}
\begin{tabbing}
x\=xxxxx\=\kill
          \>{\em exponent-part:}\\
                \>\>{\tt e} {\em sign}$_{\mbox{\scriptsize\em opt}}$ {\tt digit-sequence}\\
                \>\>{\tt E} {\em sign}$_{\mbox{\scriptsize\em opt}}$ {\tt digit-sequence}\\[2ex]
          \>{\em sign:} {\footnotesize one of}\\
                 \>\>{\tt +}\quad{\tt -}\\
\end{tabbing}
\end{quote}
In this grammar format:
\begin{itemize}
\item grammar elements are delimited by whitespace;
\item grammar elements representing terminals are written in a {\tt teletype} font;
\item grammar elements representing nonterminals are written in an {\it italic} font;
\item the first line of a new grammar rule contains a nonterminal name
concatenated with a colon {\em :} as its first element, which may
optionally be followed by the elements {\footnotesize one of} (no
other elements may appear on the line);
\item the ordinary rules, each successive line is a production; 
\item for rules using the {\footnotesize one of} construction, each
element is a production;
\item a rule is terminated by either a blank line or the start of a
new rule;
\item elements may be suffixed by $_{\mbox{\scriptsize\em opt}}$ in
which case they are optional; the production containing the optional
element is intended to be duplicated once with the optional element
and once without.
\end{itemize}

It is not hard to manually construct source files for, say, YACC from
these kinds of typeset grammars, but in practice the process is rather
error prone as humans are not very good at accurately transcribing
punctuation-like characters.  Recently produced standards are usually
available in electronic form as either Adobe Acrobat or HTML files,
and we can use the appropriate viewers to select and copy grammar
rules into a conventional text editor. The resulting ASCII files
retain character values and indentation, but suppress font information.
Here is the plain-ASCII version of the above extract.

\begin{verbatim}
     exponent-part:
         e signopt digit-sequence
         E signopt digit-sequence

     sign: one of
         +   -
\end{verbatim}

{\lt gramex} converts such fragments into a Bison-like format in which
\begin{itemize}
\item terminals are stropped with {\tt "} characters, 
\item nonterminals are
written with underscore (\verb+_+) characters instead of hyphens,
\item productions are
separated by vertical bars {\tt |} and terminated with semi-colons, and
\item  optional parts are suffixed with the {\tt ?} optional regular
operator. (Bison does not support regular expressions, but the
accompanying tool {\lt gramconv} may be used to translate regular
expressions into plain BNF rules.) 
\end{itemize}
The {\lt gramex} output for this
example is:

\begin{verbatim}
     exponent_part:
       "e" sign? digit_sequence |
       "E" sign? digit_sequence ;

     sign:
       "+" | "-" ;
\end{verbatim}

\section{Other text}
Grammar rules are often embedded in other text, and some standards use
plain-English commands with the grammar to specify tricky
constructs.

Within rules, plain-English embedded commands are treated as
productions which will generate unhelpful results that must be
manually corrected.  

Lines which {\lt gramex} does not think are part of valid grammar
rules are copied to the output as comment lines prefixed by a double
slash \verb+//+. 

Occasionally, {\lt gramex} will encounter a comment line that has the
general form of a rule start. The C++ standard contains several
instances of text paragraphs that begin `Thus:', for instance. These
lines are highlighted by being prefixed with \verb+//??+ It is
important to review the whole output of a {\lt gramex} run before
submitting it to further processing: lines preceded with \verb+//??+
are particularly likely to need attention.

The \verb+-c+ option causes {\lt gramex} to suppress all comment and
blank lines from the output leaving a single blank line between the
rules.

\section{Extensions for the Java Language Specification}

The Java Language Specification documents include two grammars: a
`pedagogic' grammar sprinkled throughout the text which is intended to
support a topic-led description of the language and the `development'
grammar summarised in a late chapter which is intended to form the
basis of development tools.

The development grammar uses two extensions to the basic scheme
employed in the ANSI C and ANSI C++ standards documents.
\begin{enumerate}
\item The [ ] and $\{ \}$ brackets stand for the optional and
Kleene-closure (zero-or-many) operations.

The \verb+-e+ option forces {\lt gramex} to treat these \verb+[+,
\verb+]+, \verb+{+ and \verb+}+ characters as EBNF meta-characters
instead of tokens. This of course leaves the problem of how to
recognise literal brackets and braces in the grammar. The work-around
is to preface these literals with some unusual prefix such
as \verb+@!@+. {\lt gramex} will treat \verb+@!@[+ as just another
terminal, outputting \verb+"@!@["+. A global search and replace may then
be used in your favourite editor to convert the \verb+"@!@["+ tokens to
\verb+"["+.

\item Long productions are split over multiple lines by indenting the
continuation lines.

The {\tt -i} options causes {\tt gramex} to treat productions that are
indented by more than four characters as continuation lines.

\end{enumerate}

\section{Extracting rules from Pascal standards}

\section{Limitations}

Since {\lt gramex} is dealing with plain-ASCII grammar rules which
have been stripped of their formatting, some unavoidable ambiguities
arise. In this section we have tried to list known problems which
should be manually checked for.
\begin{enumerate}
\item When processing C-style standards
{\lt gramex} decides whether an alphanumeric grammar element is
a nonterminal or a terminal by looking at its list of valid left hand
sides. In the ANSI C++ grammar, for instance, there is a rule for
nonterminal {\em operator:}, but {\tt operator} is also a keyword of
C so grammar element \fbox{operator} will always be interpreted as a
nonterminal. For Pascal standards, the stropping conventions allow
terminal and nonterminals to be directly recognised, so this problem
does not arise.

\item When extracting grammars using the {\tt -e} option, instances of
\verb+[+, \verb+]+, \verb+{+ and \verb+}+ elements will always be
treated as EBNF meta-characters.

\item When extracting grammars using the {\tt -i} option, productions
that are indented more than four spaces will be treated as
continuation lines.

\item When extracting grammars using the {\tt -i} option, a maximum of
one continuation line is allowed per production, that is, no
production may span more than two lines.
\end{enumerate}


\section{Implementation}
{\lt gramex} is written in ANSI-C and may be compiled with standard C
and C++ compilers. The main data structures are the input {\tt buffer}
(declared line 26 and initialised in lines 215--249) and the {\tt
lines} array which is declared on lines 28--34 and created at line
256.

The basic approach is to load the entire input
into {\tt buffer}, count the number of lines in {\tt linecount} and
then create {\tt lines}, initialising the {\tt start} fields to point
to the first character in each line.

Each line is annotated with a {\em kind}, such as blank, comment, rule
start and so on. Kinds are represented by elements of the {\tt enum}
declared in lines 23--24.

Line kinds are computed in lines 271--350 in a series of passes. The
output is produced by walking the lines array and outputting each line
of input under the control of a {\tt switch} statement that tests the
line kind at lines 358--416.

{\footnotesize
\listinginput{1}{gramex.c}
}
\bibliographystyle{alpha}
\bibliography{adrian}
\end{document}
